\chapter{Technologie lijst}
\label{ch:technologylist}

Een lijst van technologieën die gebruik maken van \textit{WebGPU} werd verzameld om een inschatting te maken hoever \textit{WebGPU} reeds toegepast wordt. Er werd vooral gezocht naar webapplicaties die gebruik maken van de inferentie rekenkracht die \textit{WebGPU} mogelijk maakt. Ook werd de testbaarheid van de verzamelde technologieën bekeken zodat deze indien mogelijk, verder konden worden onderzocht.

\section{web-llm}

Veel huidige implementaties, waarbij de rekenkracht van \textit{WebGPU} wordt ingezet voor \textit{large language models} ondersteuning in de browser, maken gebruik van \href{https://github.com/mlc-ai/web-llm}{GitHub.com/mlc-ai/web-llm}, dit is een \textit{open source} project onder de \textit{Apache License, Version 2.0}.

\bigbreak{}

\textit{Web-llm} laat toe dat \textit{large language models} direct beschikbaar zijn in de browser door middel van WebGPU. Het is een Javascript implementatie, en de software is beschikbaar als een \textit{npm} package. Het project werd gestart om meer diversiteit mogelijk te maken op het web hoe er gebruik kan gemaakt worden van \textit{LLM's} ecosysteem \autocite{mlcai2024}.

\begin{displayquote}[{\cite{mlcai2023}}]
    "WebLLM is fully compatible with OpenAI API. That is, you can use the same OpenAI API on any open source models locally, with functionalities including json-mode, function-calling, streaming, etc."
\end{displayquote}

De toegankelijkheid van deze technologie werd gedemonstreerd met een \textit{Proof of Concept} in sectie \ref{sec:chatgpu}.

\break{}

\section{embd}

embd van Fleetwood~\autocite{Fleetwood2023c}.

\section{Whisper Turbo}

\textit{Whisper turbo} en de nieuwere implementatie \textit{Ratchet Whisper} maken gebruik van het bekende Whisper model om aan de hand van \textit{WebGPU} transcriptie uit te voeren. Dit zorgt ervoor dat deze implementaties een veel gebruiksvriendelijker alternatief zijn dan andere lokale implementaties van Whisper waarvoor installatie van \textit{Python} en al dan niet \textit{CUDA} voor grafische acceleratie, vereist zijn. 

\bigbreak{}

Whisper werd in detail getest in sectie \ref{sec:whispertest}. Hiervoor werd een \textit{Python} test script geschreven dat beschikbaar is sectie \ref{whispertestcode} in de bijlage. Dit script test de uitvoeringstijd van zowel processor als \textit{CUDA} implementaties van Whisper. De \textit{WebGPU} implementatie van \textcite{Fleetwood2024} werd hiervoor ook getest en daarna vergeleken met de traditionele methoden die eerder werden vermeld.

\section{Wgpu-bench}

Wgpu-bench van Fleetwood~\autocite{Fleetwood2023d}.

\section{Markdown editor}

Markdown editor van Nico Martin, maakt gebruik van AI-modellen om de functionaliteit van deze \textit{progressive web applicatie} uit te bereiden. Markdown editor is een tekstverwerker die toestaat om spraak om te zetten in tekst, dit aan de hand van verschillende versies van \textcite{radford2022whisper}. Ook worden {large language models} ingezet om delen van tekst te verbeteren of zelfs te vertalen. 

\bigbreak{}

De parallelle rekenkracht van de grafische kaart laat toe dat de browser zelf niet wordt onderbroken. Hierdoor kan de gebruiker ongestoord verder werken, en dit maakt de applicatie zeer gebruiksvriendelijk.

\bigbreak{}

Indien deze implementatie geen gebruik zou maken van \textit{WebGPU} zou dit een merkbaar verschil geven op vlak van de snelheid, en dus ook de gebruiksvriendelijkheid. De browser zou op dat moment namelijk gebruik moeten maken van het hoofdproces. En hierdoor dus de  rekencapaciteit moeten verdelen tussen de webapplicatie en de actieve AI-modellen~\autocite{Martin2020}.

\break{}

\section{webgpu-embedding-benchmark}

Deze test van \textcite{Lochner2024} staat toe om de vergelijking te maken tussen de prestaties van \textit{WebGPU} en \textit{WASM} op vlak van \textit{embedding}, een process in \textit{machine learning} waarbij objecten zoals woorden, afbeeldingen, audio bestanden of videos kunnen worden verwerkt zodat deze dat op vlak van gelijkheid kunnen worden doorzocht. Dit is dus een belangrijke functionaliteit binnen \textit{machine learning} en computercomponenten die deze processen op een performante manier kunnen uitvoeren, laten toe om op een efficiënte manier AI-modellen te trainen~\autocite{Cloudflare2024}. Een gedetailleerde uitvoering van deze test kan gevonden worden in sectie \ref{sec:transformerbench}.

\section{Conway's Game of Life}

Omdat Conway's Game of Life simpele regels bevat die toelaten een voorbeeld implementatie makkelijk uit te voeren, maakt deze simulatie een geschikte test om de WebGPU technologie te onderzoeken. Conway's Game of Life heeft namelijk zowel een computationele als grafische aspecten. Hierdoor kunnen zowel \textit{compute shaders} als \textit{vertex shader} worden gecombineerd. De implementatie van Conway's Game of Life kan gevonden worden in sectie \ref{sec:gool}.