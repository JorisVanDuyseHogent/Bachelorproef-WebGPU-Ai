%%=============================================================================
%% Samenvatting
%%=============================================================================

% TODO: De "abstract" of samenvatting is een kernachtige (~ 1 blz. voor een
% thesis) synthese van het document.
%
% Een goede abstract biedt een kernachtig antwoord op volgende vragen:
%
% 1. Waarover gaat de bachelorproef?
% 2. Waarom heb je er over geschreven?
% 3. Hoe heb je het onderzoek uitgevoerd?
% 4. Wat waren de resultaten? Wat blijkt uit je onderzoek?
% 5. Wat betekenen je resultaten? Wat is de relevantie voor het werkveld?
%
% Daarom bestaat een abstract uit volgende componenten:
%
% - inleiding + kaderen thema
% - probleemstelling
% - (centrale) onderzoeksvraag
% - onderzoeksdoelstelling
% - methodologie
% - resultaten (beperk tot de belangrijkste, relevant voor de onderzoeksvraag)
% - conclusies, aanbevelingen, beperkingen
%
% LET OP! Een samenvatting is GEEN voorwoord!

%%---------- Nederlandse samenvatting -----------------------------------------
%
% TODO: Als je je bachelorproef in het Engels schrijft, moet je eerst een
% Nederlandse samenvatting invoegen. Haal daarvoor onderstaande code uit
% commentaar.
% Wie zijn bachelorproef in het Nederlands schrijft, kan dit negeren, de inhoud
% wordt niet in het document ingevoegd.

\IfLanguageName{english}{%
\selectlanguage{dutch}
\chapter*{Samenvatting}
\selectlanguage{english}
}{}

%%---------- Samenvatting -----------------------------------------------------
% De samenvatting in de hoofdtaal van het document

\chapter*{\IfLanguageName{dutch}{Samenvatting}{Abstract}}

De opkomst van kunstmatige intelligentie zoals \textit{large language models} beschikbaar voor het breder publiek, heeft ertoe geleid dat de nood aan \textit{server-side} rekenkracht ontzettend is toegenomen. Deze groei zal doorzetten naarmate de technologie verder wordt uitgewerkt. Het is van groot belang dat AI-modellen op een efficiënte manier kunnen worden ingezet, zodat zowel de kosten als de impact op het milieu laag worden gehouden. 

\bigbreak{}

Een opkomende technologie zoals \textit{WebGPU} staat toe lokale rekenkracht beschikbaar te stellen op de apparatuur van de eindgebruiker, en in te zetten binnen de browser. In dit onderzoek worden de mogelijkheden van \textit{WebGPU} beschreven en onderbouwd aan de hand van technische testen. \textit{WebGPU} heeft het potentieel om de proliferatie van kunstmatige intelligentie op een open, gebruiksvriendelijk en privacy bewust internet mogelijk te maken.

\bigbreak{}

\textit{WebGPU} laat namelijk toe inferentie, een wiskundig proces dat vereist is om AI-modellen te gebruiken, lokaal uit te voeren. Hierdoor is het een gepaste technologie om volledig of gedeeltelijk rekenkrachtintensieve taken over te nemen die voorheen uitgevoerd werden door een server. Hiervoor moeten AI-modellen echter lokaal beschikbaar gesteld worden, dit kan een struikelblok vormen voor implementaties van \textit{WebGPU}.

\bigbreak{}

Webapplicaties, die echter versterkt worden door kunstmatige intelligentie en die daarnaast ook nog gebruik maken van lokale componenten, via \textit{WebGPU}, zouden kunnen bijdragen aan een snel en gebruiksvriendelijk internet. Voor een eindgebruiker met sterke apparatuur betekent dit dat een applicatie met deze complexe software lokaal beschikbaar wordt, en makkelijk te installeren is. Voor online bedrijven houdt dit in dat er een efficiënte oplossing bestaat om AI-tech\-no\-lo\-gieën te kunnen voorzien met minimale implementatie complexiteit en met een zeer lage onderhoudskosten.

\bigbreak{}

Binnen dit onderzoek wordt \textit{WebGPU} gecombineerd met \textit{large language models} als \textit{proof of concept}. Ook wordt er onderzocht hoe de performantie van \textit{WebGPU} zich verhoudt tot reeds bestaande technologieën. De combinatie van deze test-op\-ste\-llingen geeft een duidelijk beeld over de bruikbaarheid van \textit{WebGPU} als technologie ter ondersteuning van kunstmatige intelligentie op het web.