%%=============================================================================
%% Methodologie
%%=============================================================================

\chapter{\IfLanguageName{dutch}{Methodologie}{Methodology}}%
\label{ch:methodologie}

%% TODO: In dit hoofdstuk geef je een korte toelichting over hoe je te werk bent
%% gegaan. Verdeel je onderzoek in grote fasen, en licht in elke fase toe wat
%% de doelstelling was, welke deliverables daar uit gekomen zijn, en welke
%% onderzoeksmethoden je daarbij toegepast hebt. Verantwoord waarom je
%% op deze manier te werk gegaan bent.
%% 
%% Voorbeelden van zulke fasen zijn: literatuurstudie, opstellen van een
%% requirements-analyse, opstellen long-list (bij vergelijkende studie),
%% selectie van geschikte tools (bij vergelijkende studie, "short-list"),
%% opzetten testopstelling/PoC, uitvoeren testen en verzamelen
%% van resultaten, analyse van resultaten, ...
%%
%% !!!!! LET OP !!!!!
%%
%% Het is uitdrukkelijk NIET de bedoeling dat je het grootste deel van de corpus
%% van je bachelorproef in dit hoofdstuk verwerkt! Dit hoofdstuk is eerder een
%% kort overzicht van je plan van aanpak.
%%
%% Maak voor elke fase (behalve het literatuuronderzoek) een NIEUW HOOFDSTUK aan
%% en geef het een gepaste titel.

Omdat de ontwikkeling van WebGPU plaats vindt in een open domein, is het vergaren van informatie een belangrijke maar in het algemeen een toegankelijke opdracht. Het inlezen in de technologie achter \textit{WebGPU} dat vooraf ging aan dit onderzoek was echter cruciaal om de praktische zijde verder te zetten. Hierdoor kon dit aspect worden onderbouwd met technische implementaties en konden er conclusies getrokken worden rond zaken die reeds door anderen werden voorgelegd. Al is de ontwikkeling van \textit{WebGPU} al reeds een tijd actief, de globale opname van deze technologie door ontwikkelaars en software bedrijven verloopt op een ander tempo. 

\bigbreak{}

De wereld die ontstaat rond \textit{WebGPU} is dus nog nieuw en klein. Het is echter belangrijk op te merken dat grote bedrijven deze technologie ondersteunen en er actief aan meewerken. De documentatie die actief wordt opgesteld voor \textit{WebGPU} door \textcite{W3C2023} laat toe om complexe uitwerkingen en details van \textit{WebGPU} te onderzoeken. Ook is het belangrijk dat ontwikkelaars die de technologie hebben vast gegrepen en er actief onderzoek achter doen te blijven volgen. Het publiceren van nieuwe informatie verloopt namelijk met een vertraging, maar onderzoek naar deze nieuwe technologie gebeurd voortdurend en levert interessante resultaten. Omwille van deze zaken die eigen zijn aan nieuwe technologieën werd tijdens het uitvoeren van het onderzoek opnieuw teruggekeerd naar ontwikkelaars zodat vooruitgang kon worden opgevolgd.

\break{}

\section{Introductie tot WebGPU en shaders}

De eerste, stap nadat het theoretische onderzoek werd afgerond, was het uitwerken van een simpele demonstratie. Uit de literatuurstudie bleek dat het uitwerken van complexe zaken een te grote stap was. Een implementatie van \textit{Conway's Game of Life} was echter wel geschikt als introductie. Omdat deze 
\textit{zero-player game} zowel een grafisch als computationeel aspect heeft kan \textit{WebGPU} worden ingezet om de simulatie volledig uit te voeren~\autocite{google2023}.

\section{Technologie lijst opstellen}

Het toegankelijk maken van kunstmatige intelligentie via het web is een grote uitdaging, die tot nu toe gepaard gaat met installatie van complexe software en hoge hardware vereisten, indien dit wordt uitgevoerd op de apparatuur van eind-gebruikers. \textit{WebGPU} leent zich ertoe dit process te vereenvoudigen. Er wordt onderzocht op welke manieren \textit{WebGPU} toelaat het process gebruiksvriendelijker te maken. 

\bigbreak{}

Verzamelen van reeds bestaande technologieën die gebruik maken van \textit{WebGPU}. Hierbij is er ook een mogelijkheid om te beschrijven waartoe deze softwareoplossingen in staat zijn, en hoe deze al dan niet toepasbaar en relevant zijn voor het onderzoek. Na het verzamelen van geschikte technologieën, wordt er een selectie gemaakt van software en testen die verder voor het onderzoek zullen worden uitgewerkt. Hierdoor kunnen zaken verder in detail worden uitgewerkt en kan de focus en het domein van het onderzoek beperkt worden.

\section{Performantie testen}

Door de performantie van \textit{WebGPU} te vergelijken met bestaande technologieën kan er een beeld worden gevormd hoever de technologie staat. Hierdoor kan er ook bepaald worden of WebGPU geschikt is voor verdere toepassingen zoals het trainen van kunstmatige intelligentie. Er wordt een vergelijking gemaakt met \textit{CPU} prestaties in de vorm van \textit{WASM}. Maar ook met andere \textit{GPU API's} zoals \textit{CUDA}. Ook wordt WebGPU vergeleken met voorgangers zoals \textit{WebGL}.

\section{Proof of Concept}

Om de mogelijkheden en performantie van \textit{WebGPU} te kunnen waarnemen worden verschillende testopstellingen opgezet. De resultaten van deze opstellingen kunnen demonstreren of \textit{WebGPU} een geschikte kandidaat is om kunstmatige intelligentie toegankelijker te maken voor het grote publiek. Door simpele implementaties te kunnen presenteren aan de hand van \textit{Proof of Concept} opstellingen kan een antwoord op de onderzoeksvragen worden geboden. Hierdoor kan er ook op een gebruiksvriendelijke manier met de technologie kennis gemaakt worden.

\section{Conclusie}

Nadat zowel de performantie testen en \textit{Proof of Concept} opstellingen zijn uitgevoerd kan er een beeld geschept worden over de huidige staat van WebGPU. Hierdoor kunnen aanbevelingen gemaakt worden over de technologie.
