%%=============================================================================
%% Methodologie
%%=============================================================================

\chapter{\IfLanguageName{dutch}{Methodologie}{Methodology}}%
\label{ch:methodologie}

%% TODO: In dit hoofdstuk geef je een korte toelichting over hoe je te werk bent
%% gegaan. Verdeel je onderzoek in grote fasen, en licht in elke fase toe wat
%% de doelstelling was, welke deliverables daar uit gekomen zijn, en welke
%% onderzoeksmethoden je daarbij toegepast hebt. Verantwoord waarom je
%% op deze manier te werk gegaan bent.
%% 
%% Voorbeelden van zulke fasen zijn: literatuurstudie, opstellen van een
%% requirements-analyse, opstellen long-list (bij vergelijkende studie),
%% selectie van geschikte tools (bij vergelijkende studie, "short-list"),
%% opzetten testopstelling/PoC, uitvoeren testen en verzamelen
%% van resultaten, analyse van resultaten, ...
%%
%% !!!!! LET OP !!!!!
%%
%% Het is uitdrukkelijk NIET de bedoeling dat je het grootste deel van de corpus
%% van je bachelorproef in dit hoofdstuk verwerkt! Dit hoofdstuk is eerder een
%% kort overzicht van je plan van aanpak.
%%
%% Maak voor elke fase (behalve het literatuuronderzoek) een NIEUW HOOFDSTUK aan
%% en geef het een gepaste titel.

Omdat de ontwikkeling van \textit{WebGPU} plaats vindt in een open domein is het vergaren van informatie een belangrijke, en in het algemeen een toegankelijke opdracht. Het inlezen in de technologie achter \textit{WebGPU}, dat vooraf ging aan dit onderzoek, was echter cruciaal om de praktische kant verder te zetten. Hierdoor kon dit aspect worden onderbouwd met technische implementaties en konden er conclusies getrokken worden uit zaken die reeds door anderen werden voorgelegd. De ontwikkeling van \textit{WebGPU} is reeds een tijd bezig, echter de globale opname van deze technologie door ontwikkelaars en software bedrijven verloopt op een langzamer tempo. 

\bigbreak{}

De wereld die ontstaat rond \textit{WebGPU} is nog nieuw en klein. Het is echter belangrijk op te merken dat grote bedrijven deze technologie ondersteunen en er actief aan meewerken. De participatie van deze bedrijven kan worden bekeken in de \textit{GPU for the Web Working Group} waarbij vertegenwoordigers van bedrijven zoals \textit{Microsoft}, \textit{Google}, \textit{Apple}, \textit{Intel} en de \textit{Mozilla Foundation} deelnemen \autocite{W3C2024a}. De documentatie die actief wordt opgesteld voor \textit{WebGPU} door \textcite{W3C2023} laat toe om complexe uitwerkingen en details van \textit{WebGPU} te onderzoeken. 

\bigbreak{}

Ook is het belangrijk om ontwikkelaars die onderzoek doen naar deze nieuwe technologie te blijven opvolgen. Het publiceren van nieuwe informatie verloopt namelijk met een vertraging, maar onderzoek is continu en levert interessante resultaten op. Omwille van deze zaken, die eigen zijn aan nieuwe technologieën, werd tijdens het uitvoeren van dit onderzoek opnieuw teruggekeerd naar ontwikkelaars, zodat vooruitgang kon worden opgevolgd.

\section{Introductie tot WebGPU en shaders}

De eerste stap nadat het theoretische onderzoek wordt afgerond, is het uitwerken van een simpele demonstratie. Een implementatie van \textit{Conway's Game of Life} aan de hand van \textit{WebGPU} is geschikt als introductie. Omdat deze 
\textit{zero-player game} zowel een grafisch als computationeel aspect heeft kan \textit{WebGPU} worden ingezet om de simulatie volledig uit te voeren.

\section{Technologielijst opstellen}

Het toegankelijk maken van kunstmatige intelligentie via het web is een grote uitdaging. Lokale implementaties vereisen installatie van complexe software. \textit{WebGPU} leent zich ertoe dit process te vereenvoudigen. Bij het verzamelen van technologieën wordt in dit hoofdstuk onderzocht hoe \textit{WebGPU} dit process ge\-bruiks\-vrien\-de\-lij\-ker kan maken. Hierbij wordt er beschreven waartoe deze softwareoplossingen in staat zijn. Na het verzamelen van geschikte technologieën wordt een selectie gemaakt van software die verder voor het onderzoek zal worden uitgewerkt. Door deze selectie verder in detail te bespreken wordt de focus en het domein van het onderzoek beperkt gehouden.

\section{Performantie testen}

Door de performantie van \textit{WebGPU} te vergelijken met bestaande technologieën kan er een beeld worden gevormd hoever de technologie staat. Hierdoor kan er ook bepaald worden of \textit{WebGPU} geschikt is voor verdere toepassingen zoals het trainen van kunstmatige intelligentie. Er wordt een vergelijking gemaakt met \textit{CPU} prestaties in de vorm van \textit{WASM}. Maar ook met andere \textit{GPU API's} zoals \textit{CUDA}. Ook wordt \textit{WebGPU} vergeleken met voorgangers zoals \textit{WebGL}.

\section{Proof of Concept}

Om de mogelijkheden en performantie van \textit{WebGPU} te kunnen waarnemen wordt een prototype uitgewerkt. De resultaten van deze opstellingen kunnen demonstreren of \textit{WebGPU} een geschikte kandidaat is om lokale kunstmatige intelligentie toegankelijker te maken. Door deze simpele implementaties te kunnen presenteren  kan een antwoord worden gegeven op de onderzoeksvragen. Hierdoor kan er ook op een gebruiksvriendelijke manier met de technologie kennis gemaakt worden.

\section{Conclusie}

Nadat zowel de performantie testen en \textit{Proof of Concept} opstellingen zijn uitgevoerd kan er een beeld geschept worden over de huidige staat van WebGPU. Hierdoor kunnen aanbevelingen gemaakt worden over de technologie.
