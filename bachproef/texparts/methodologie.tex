%%=============================================================================
%% Methodologie
%%=============================================================================

\chapter{\IfLanguageName{dutch}{Methodologie}{Methodology}}%
\label{ch:methodologie}

%% TODO: In dit hoofdstuk geef je een korte toelichting over hoe je te werk bent
%% gegaan. Verdeel je onderzoek in grote fasen, en licht in elke fase toe wat
%% de doelstelling was, welke deliverables daar uit gekomen zijn, en welke
%% onderzoeksmethoden je daarbij toegepast hebt. Verantwoord waarom je
%% op deze manier te werk gegaan bent.
%% 
%% Voorbeelden van zulke fasen zijn: literatuurstudie, opstellen van een
%% requirements-analyse, opstellen long-list (bij vergelijkende studie),
%% selectie van geschikte tools (bij vergelijkende studie, "short-list"),
%% opzetten testopstelling/PoC, uitvoeren testen en verzamelen
%% van resultaten, analyse van resultaten, ...
%%
%% !!!!! LET OP !!!!!
%%
%% Het is uitdrukkelijk NIET de bedoeling dat je het grootste deel van de corpus
%% van je bachelorproef in dit hoofdstuk verwerkt! Dit hoofdstuk is eerder een
%% kort overzicht van je plan van aanpak.
%%
%% Maak voor elke fase (behalve het literatuuronderzoek) een NIEUW HOOFDSTUK aan
%% en geef het een gepaste titel.

\section{literatuurstudie}

De wereld van WebGPU is heel nieuw en klein, het is echter belangrijk op te merken dat heel grote bedrijven deze technologie ondersteunen en er actief aan mee werken. De documentatie die werd opgesteld voor WebGPU door \textcite{W3C2023} laat toe om complexe uitwerkingen en details van \textit{WebGPU}

\section{Introductie tot WebGPU en shaders}

De eerste stap nadat het theoretische onderzoek werd afgerond was het uitwerken van een simpele demonstratie. Uit de literatuurstudie bleek dat het uitwerken van complexe zaken een te grote stap was. Een implementatie van \textit{Conway's Game of Life} was echter wel geschikt als introductie. Omdat deze 
\textit{zero-player game} zowel een grafisch als computationeel aspect heeft kan \textit{WebGPU} worden ingezet om de simulatie volledig uit te voeren. \autocite{google2023}

\section{Requirement analyse}

Het toegankelijk maken van kunstmatige intelligentie via het web is een grote uitdaging die tot nu toe gepaard gaat met installatie van complexe software en hoge hardware vereisten indien dit uitgevoerd op de apparatuur van eind-gebruikers. WebGPU leent zich ertoe dit process te simplificeren, 

\section{long list van technologieën}

Verzamelen van reeds bestaande technologieën en beschrijven waartoe deze in staat zijn.

\section{Short list van technologieën}



\section{Performantie testen}

Door de performantie van WebGPU te vergelijken met bestaande technologieën kan er een beeld worden gevormd hoever te technologie staat. Hierdoor kan er ook bepaald worden of WebGPU geschikt is voor verdere toepassingen zoals het trainen van kunstmatige intelligentie.

\section{Proof of Concept}

Om de mogelijkheden en performantie van \textit{WebGPU} te kunnen waarnemen worden verschillende testopstellingen opgezet en getest. De resultaten van deze worden in een opstellingen kunnen demonstreren of \textit{WebGPU} een geschikte kandidaat is om kunstmatige intelligentie toegankelijker te maken voor het grote publiek.

\section{Conclusie}


