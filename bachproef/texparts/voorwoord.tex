%%=============================================================================
%% Voorwoord
%%=============================================================================

\chapter*{\IfLanguageName{dutch}{Woord vooraf}{Preface}}%
\label{ch:voorwoord}

%% TODO:
%% Het voorwoord is het enige deel van de bachelorproef waar je vanuit je
%% eigen standpunt (``ik-vorm'') mag schrijven. Je kan hier bv. motiveren
%% waarom jij het onderwerp wil bespreken.
%% Vergeet ook niet te bedanken wie je geholpen/gesteund/... heeft

In de loop van de voorbije jaren groeide mijn fascinatie voor grafische rekenkracht enorm, inclusief alles wat hier mee te maken had. Van jongs af aan begon dit simpelweg met het spelen van videogames, zoals vele jongeren. Dit groeide later uit tot het bouwen van eigen computers en het vergelijken van prestaties. De grootste uitdaging was om met zo weinig mogelijk apparatuur zoveel mogelijk te bereiken. Dat gaf mij de grootste voldoening.

\bigbreak{}

Mijn interesse in computers zette zich voort in het uitbouwen van \textit{crypto mining} installaties. Deze apparatuur kwam goed van pas toen ik mijn studie informatica aan de HOGENT startte. Een lector cybersecurity daagde me toen uit om wachtwoorden te kraken voor een labo. Hierbij is rekenkracht de oplossing. Het waren deze zaken die mij er toe leidden de rekenkracht van deze com\-pu\-ter\-on\-der\-de\-len te herkennen, maar vooral te respecteren. 

\bigbreak{}

Ik dacht dat ik tijdens mijn verdere opleiding niet meer in aanraking zou komen met de hardwarewereld. Ik had namelijk gekozen voor de specialisatie Mobile \& Enterprise developer. Dit vond ik uiteraard jammer omdat het toch zaken zijn die mij ontzettend boeien. Het was pas toen ik de \textit{WebGPU} technologie ontdekte, en hierdoor opnieuw gefascineerd raakte met grafische en algemene rekenkracht, dat voor mij deze wereld terug open ging.

\bigbreak{}

\textit{WebGPU} wekte mijn interesse en dus besloot ik mijn bachelorproef over deze technologie te schrijven. Dat bleek een behoorlijke uitdaging vanwege mijn gelimiteerde kennis rond \textit{GPU}-technologie en kunstmatige intelligentie. Hierbij kreeg ik ontzettend veel steun van mijn vriendin die, al is dit niet haar specialisatie, toch altijd geïnteresseerd luisterde naar mijn hersenspinsels en meestal constructieve feedback kon geven over mijn ideeën.

\bigbreak{}

Het uitvoeren van dit onderzoek en schrijven van deze bachelorproef werd mogelijk gemaakt door de hulp, motivatie en het vertrouwen van mijn promotor Dhr. Desmedt, maar ook mijn ouders, kotgenoten en mijn vriendin. Voor hun geduld doorheen deze periode ben ik ze ontzettend dankbaar. Ook heeft dit ervoor gezorgd dat ik een grote groep mensen had waarop ik kon steunen. Hierbij hebben ze mij ook toegestaan en ondersteund om mijn eigen pad te kiezen en mijn koppige aard een kans te gegeven. Hieruit volgde een onderzoek over een technologie die mij ontzettend interesseert.