%%=============================================================================
%% Voorwoord
%%=============================================================================

\chapter*{\IfLanguageName{dutch}{Woord vooraf}{Preface}}%
\label{ch:voorwoord}

%% TODO:
%% Het voorwoord is het enige deel van de bachelorproef waar je vanuit je
%% eigen standpunt (``ik-vorm'') mag schrijven. Je kan hier bv. motiveren
%% waarom jij het onderwerp wil bespreken.
%% Vergeet ook niet te bedanken wie je geholpen/gesteund/... heeft

In de loop van de voorbije jaren groeide mijn fascinate voor grafische computatie enorm, dus ook alles wat hier mee te maken had. Van jongs af aan begon dit simpelweg met interesse in gaming, later groeide dit uit tot het bouwen van eigen computers en het vergelijkenvan performantie. De grootste uitdaging en het grootste genoegen was voor mij om met weinig apparatuur veel te kunnen doen. 

\bigbreak{}

Dit deel van mijn leven werd mogelijk gemaakt door mijn ouders hun steun en toevertrouwen, zij gaven mijn de kans om met technologie te spelen, en hieruit een leer process op te starten. Echter moet ik hierbij vermelden dat de leerfactor in het prille begin nog redelijk laag lag. Mijn ouders hebben altijd het geduld gehad om mij, op mijn eigen manier, de wereld te laten ontdekken en te leren wat allemaal mogelijk is, hiervoor ben ik ze ontzettend dankbaar.

\bigbreak{}

Mijn interesse in computers zette zich voort in het uitbouwen van crypto mining installaties. Deze installaties kwamen goed van pas toen ik later mijn Web-ontwikkeling specialisatie aan de HOGENT startte. Een lector cybersecurity daagde me toen uit om wachtwoorden te kraken voor een labo, hierbij is rekenkracht de oplossing. Het was het verloop van deze zaken die mij er toe leiden de rekenkracht van deze computer onderdelen te herkennen maar vooral te respecteren. 

\bigbreak{}

Verder dacht ik dat ik tijdens deze opleiding niet meer in aanraking zou komen met de hardware wereld omwille van de ontelbare abstracte lagen die liggen tussen het intrageren met een web applicatie en de onderliggende technologie.

\bigbreak{}

Het was pas toen ik onder andere door Dustin Brett zijn indrukwekkende webapplicatie \textit{daedalOS} de WebGPU technologie ontdekte, en hierdoor opnieuw gefascineerd raakte met grafische en algemene computatie in de browser dat voor mij deze wereld terug open ging.

\bigbreak{}

Voor mij was toen het verhaal pas echt begonnen, en dus besloot ik mijn bachelorproef over de WebGPU technologie te schrijven. Ik had eerder wel eens een video spel proberen schrijven met een kotgenoot en jeugdvriend in Unity maar ik had geen voorkennis over hoe GPU APIs precies in elkaar zitten en al minder hoe shaders worden geprogrammeerd, ik kwam dus terecht in een hele
nieuwe facineerden maar uiteraard ook uitdagende en complexe wereld terecht.