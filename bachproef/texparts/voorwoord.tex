%%=============================================================================
%% Voorwoord
%%=============================================================================

\chapter*{\IfLanguageName{dutch}{Woord vooraf}{Preface}}%
\label{ch:voorwoord}

%% TODO:
%% Het voorwoord is het enige deel van de bachelorproef waar je vanuit je
%% eigen standpunt (``ik-vorm'') mag schrijven. Je kan hier bv. motiveren
%% waarom jij het onderwerp wil bespreken.
%% Vergeet ook niet te bedanken wie je geholpen/gesteund/... heeft

In de loop van de voorbije jaren groeide mijn fascinate voor grafische computatie enorm, inclusief alles wat hier mee te maken had. Van jongs af aan begon dit simpelweg met interesse in gaming wat later uit groeide tot het bouwen van eigen computers en het vergelijken van performantie. De grootste uitdaging en het grootste genoegen was voor mij om met weinig apparatuur veel te kunnen bereiken. 

\bigbreak{}

Deze fascinatie kon ik verder ontwikkelen met steun en vertrouwen van mijn ouders. Zij gaven mij de kans om met technologie te spelen, en hieruit een leer process op te starten. Echter was de leerfactor in het begin nog redelijk laag. Mijn ouders hebben steeds het geduld gehad om mij op mijn eigen manier de wereld te laten ontdekken en te leren wat allemaal mogelijk is. Hiervoor ben ik ze ontzettend dankbaar.

\bigbreak{}

Mijn interesse in computers zette zich verder in het uitbouwen van \textit{crypto mining} installaties. Deze apparatuur kwam goed van pas toen ik mijn studie informatica aan de HOGENT startte. Een lector cybersecurity daagde me toen uit om wachtwoorden te kraken voor een labo. Hierbij is rekenkracht de oplossing. Het waren deze zaken die mij er toe leiden de rekenkracht van deze computer onderdelen te herkennen, maar vooral te respecteren. 

\bigbreak{}

Ik dacht dat ik tijdens mijn verdere opleiding niet meer in aanraking zou komen met de hardware wereld. Ik had namelijk gekozen voor de specialisatie Mobile \& Enterprise developer. Dit vond ik uiteraard jammer omdat het toch zaken zijn die mij ontzettend boeien.

\bigbreak{}

Het was pas toen ik onder andere door Dustin Brett zijn indrukwekkende webapplicatie \textit{daedalOS} op \href{https://dustinbrett.com/}{dustinbrett.com}, de \textit{WebGPU} technologie ontdekte, en hierdoor opnieuw gefascineerd raakte met grafische en algemene rekenkracht, dat voor mij deze wereld terug open ging.

\bigbreak{}

Dat wekte mijn interesse, en dus besloot ik mijn bachelorproef over \textit{WebGPU} technologie te schrijven. Dat bleek een behoorlijke uitdaging vanwege van mijn gelimiteerde kennis over \textit{GPU} technologie en kunstmatige intelligentie. Hierbij kreeg ik ontzettend veel steun van mijn vriendin die, al is het haar specialisatie niet, toch altijd geïnteresseerd luisterde naar mijn hersenspinsels en meestal constructieve feedback kon geven over mijn ideeën.