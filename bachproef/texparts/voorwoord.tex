%%=============================================================================
%% Voorwoord
%%=============================================================================

\chapter*{\IfLanguageName{dutch}{Woord vooraf}{Preface}}%
\label{ch:voorwoord}

%% TODO:
%% Het voorwoord is het enige deel van de bachelorproef waar je vanuit je
%% eigen standpunt (``ik-vorm'') mag schrijven. Je kan hier bv. motiveren
%% waarom jij het onderwerp wil bespreken.
%% Vergeet ook niet te bedanken wie je geholpen/gesteund/... heeft

In de loop van de voorbije jaren groeide mijn fascinate voor grafische computatie enorm, dus ook alles wat hier mee te maken had. Van jongs af aan begon dit simpelweg met interesse in gaming, later groeide dit uit tot het bouwen van eigen computers en het vergelijken van performantie. De grootste uitdaging en het grootste genoegen was voor mij om met weinig apparatuur veel te kunnen bereiken. 

\bigbreak{}

Deze fascinatie kon ik blijven voort zetten door mijn ouders hun steun en toevertrouwen, zij gaven mijn de kans om met technologie te spelen, en hieruit een leer process op te starten. Echter moet ik hierbij vermelden dat de leerfactor in het begin nog redelijk laag lag. Mijn ouders hebben altijd het geduld gehad om mij, op mijn eigen manier, de wereld te laten ontdekken en te leren wat allemaal mogelijk is, hiervoor ben ik ze ontzettend dankbaar.

\bigbreak{}

Mijn interesse in computers zette zich voort in het uitbouwen van \textit{crypto mining} installaties. Deze installaties kwamen goed van pas toen ik later mijn studie informatica aan de HOGENT startte. Een lector cybersecurity daagde me toen uit om wachtwoorden te kraken voor een labo, hierbij is rekenkracht de oplossing. Het was het verloop van deze zaken die mij er toe leiden de rekenkracht van deze computer onderdelen te herkennen maar vooral te respecteren. 

\bigbreak{}

Verder dacht ik dat ik tijdens deze opleiding niet meer in aanraking zou komen met de hardware wereld. Ik had namelijk gekozen voor de specialisatie Mobile \& Enterprise developer. Dit vond ik uiteraard jammer omdat het toch zaken zijn die mij ontzettend boeien.

\bigbreak{}

Het was pas toen ik onder andere door Dustin Brett zijn indrukwekkende webapplicatie \textit{daedalOS} de WebGPU technologie ontdekte, en hierdoor opnieuw gefascineerd raakte met grafische en algemene computatie in de browser dat voor mij deze wereld terug open ging.

\bigbreak{}

Voor mij was toen het verhaal pas echt begonnen, en dus besloot ik mijn bachelorproef over de WebGPU technologie te schrijven. Wat toch een grote uitdaging bleek te zijn omwille van mijn gelimiteerde kennis over GPU technologie en kunstmatige intelligentie. Hierbij kreeg ik ontzettend veel steun van mijn vriendin die, al is het haar specialisatie niet, toch altijd geïnteresseerd mijn hersenspinsels uit luisterde en meestal constructieve feedback kon geven over mijn ideeën.