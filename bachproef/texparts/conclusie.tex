%%=============================================================================
%% Conclusie
%%=============================================================================

\chapter{Conclusie}%
\label{ch:conclusie}

% TODO: Trek een duidelijke conclusie, in de vorm van een antwoord op de
% onderzoeksvra(a)g(en). Wat was jouw bijdrage aan het onderzoeksdomein en
% hoe biedt dit meerwaarde aan het vakgebied/doelgroep? 
% Reflecteer kritisch over het resultaat. In Engelse teksten wordt deze sectie
% ``Discussion'' genoemd. Had je deze uitkomst verwacht? Zijn er zaken die nog
% niet duidelijk zijn?
% Heeft het onderzoek geleid tot nieuwe vragen die uitnodigen tot verder 
%onderzoek?

% Is WebGPU een geschikte technologie om de rekenkracht die vereist is bij het operationeel inzetten van kunstmatige intelligentie op grote schaal over te dragen aan de eindgebruiker. Staat deze overdracht een verbetering toe voor de gebruikservaring en vraag naar privacy?

% Om antwoord te geven op de onderzoeksvraag die voor dit onderzoek werd gesteld; is \textit{WebGPU} een geschikte technologie overeenstemt met het potentiële evenaren of overstemmen van de huidige technologie die word onderzocht. Hierbij spelen gebruiksvriendelijkheid en de kans op meer privacy ook een rol.

% \bigbreak{}

\iffalse
TODO Samen hang moet worden verbetert, alle zaken die hier besproken zijn geweest moeten zeker aan bod zijn gekomen in het onderzoek! Al dan niet verder in detail bespreken.
\fi

Het beschikbaar stellen van AI-modellen in web-applicaties en software blijkt een bewezen interessante investering te zijn voor bedrijven, echter wordt de eindgebruiker hierbij gedwongen gebruik te maken van de externe diensten. En is privacy niet altijd gegarandeerd. Deze twee minpunten zijn te vermijden met het gebruik van \textit{WebGPU}.

\bigbreak{}

Door verschillende testen uit te voeren, werd de rekencapaciteit van \textit{WebGPU} onderzocht. Uit deze resultaten werd geconcludeerd dat \textit{WebGPU}, zoals verwacht, een hoge rekencapaciteit kan ontgrendelen die voorheen enkel beschikbaar was door \textit{lower-level APIs} te gebruiken zoals \textit{Metal}, \textit{Direct3D}, \textit{Vulcan} of \textit{OpenGL}.

\bigbreak{}

Sterker nog, \textit{WebGPU} toont niet alleen de capaciteit om  AI-modellen voor normaal gebruik te kunnen ondersteunen, maar ook kunnen trainen. Het feit dat de benodigde \textit{hardware} afhankelijk wordt van de eindgebruiker moet natuurlijk in acht worden gehouden, maar omdat voor 70\% van de internet gebruikers, WebGPU beschikbaar is laat toe deze hardware ook verder te kunnen inzetten binnen een browser omgeving. Dit wijst erop dat \textit{WebGPU} in staat is zijn om de proliferatie van lokale inferentie mogelijk te maken, en hierbij het eerste minpunt van externe diensten te overstemmen.

\bigbreak{}

Door deze conclusie te combineren met de eerder besproken toegankelijkheid van \textit{WebGPU} kan een krachtige synergie aangetoond worden die de eindgebruiker in staat zal stellen om krachtige applicaties op een gebruiksvriendelijke manier te installeren en gebruiken. Zonder afhankelijk te moeten zijn van complexe achterliggende technologieën of externe diensten. Dit maakt een implementatie van \textit{WebGPU} ontzettend aantrekkelijk voor kleinschalige softwareoplossingen die, in tegenstelling tot voormalige grafische geaccelereerde software, zeer eenvoudig te onderhouden zijn.


\bigbreak{}

Niet enkel stelt WebGPU de gebruiker in staat om deel te nemen aan de kunstmatige intelligentie \textit{gold rush}, het laat ook toe om dit op een privacy bewuste manier te doen. Dit omwille van de minimale datauitwisseling die vereist is met externe diensten. Voor de toekomstige privacy bewuste gebruiker kan \textit{WebGPU} technologie zeer waardevol blijken. Hiervoor moet echter wel verder onderzocht worden wat de impact is van WebGPU gebruikers op zaken zoals \textit{digital fingerprinting}. Ook moeten de cybersecurity aspecten van \textit{WebGPU} verder worden onderzocht, zo kan de technologie namelijk gevoelig zijn voor \textit{side channel attacks}.

\bigbreak{}

Naast het beschikbaar maken van lokale inferentie belooft \textit{WebGPU} nog veel andere innovaties voor de \textit{browsers}. De uitdaging voor \textit{browserfabrikanten} en webontwikkelaars zal zijn om van deze technologie gebruik te maken en hierdoor de \textit{WebGPU} evolutie verder voort te zetten.