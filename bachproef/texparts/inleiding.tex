%%=============================================================================
%% Inleiding
%%=============================================================================

\chapter{\IfLanguageName{dutch}{Inleiding}{Introduction}}
\label{ch:inleiding}

% De inleiding moet de lezer net genoeg informatie verschaffen om het onderwerp te begrijpen en in te zien waarom de onderzoeksvraag de moeite waard is om te onderzoeken. In de inleiding ga je literatuurverwijzingen beperken, zodat de tekst vlot leesbaar blijft. Je kan de inleiding verder onderverdelen in secties als dit de tekst verduidelijkt. Zaken die aan bod kunnen komen in de inleiding~\autocite{Pollefliet2011}:

% \begin{itemize}
%   \item context, achtergrond
%   \item afbakenen van het onderwerp
%   \item verantwoording van het onderwerp, methodologie
%   \item probleemstelling
%   \item onderzoeksdoelstelling
%   \item onderzoeksvraag
%   \item \ldots
% \end{itemize}

\section{Context en achtergrond} %custom title
Toen \textit{OpenAI ChatGPT} lanceerde voor het grotere publiek was dit een technologie die een revolutie met zich meebracht waarvan de schaal vandaag de dag nog steeds moeilijk valt in te schatten~\autocite{Marr2023}. Niet alleen \textit{large language models} maar ook andere vormen van kunstmatige intelligentie, zullen in de toekomst een grote impact hebben op de manier waarop een eindgebruiker interageert met technologie~\autocite{Shumylo2023a}.

\bigbreak{}

\textit{Browser} webtechnologie zorgt ervoor dat kunstmatige intelligentie nog toegankelijker is voor het bredere publiek omdat het geen complex in\-sta\-lla\-tie\-pro\-ces vereist. De \textit{browser} is ook een bekende omgeving voor vele gebruikers.

\bigbreak{}

De combinatie van kunstmatige intelligentie en de web browser zorgt er voor dat een complexe technologie zoals \textit{ChatGPT} op een gebruiksvriendelijke manier kan worden ingezet. Maar uiteindelijk worden deze modellen operationeel gehouden door onder andere \textit{OpenAI} in grote datacentra~\autocite{Warren2023}. Daarom moet alle informatie die nodig is om deze technologie te gebruiken over het internet worden verzonden alvorens interactie kan plaatsvinden, wat eigen is aan webtechnologie.

\bigbreak{}

Het installeren van kunstmatige intelligentie bij de eindgebruiker is een complexe materie omdat hiervoor niet alleen \textit{drivers}, maar ook grafische interfaces vereist zijn. Ook is de apparatuur van eindgebruikers uiterst divers waardoor een universele softwareoplossing die makkelijk te onderhouden valt, moeilijk is waar te maken. Nieuwe webtechnologieën zoals \textit{Progressive Web Apps} (\textit{PWAs}) laten toe dat een gebruiker heel gemakkelijk web applicaties kan installeren zonder dat hiervoor een complex installatieproces moet doorlopen worden~\autocite{Pekala2023}. Een apparaat dat moderne browser ondersteuning heeft, biedt in principe de mogelijkheid om een \textit{Progressive Web App} te installeren. Dit komt omdat de onderliggende software niet zoveel verschilt~\autocite{Todavchich2019}. \textit{PWA}'s maken namelijk gebruik van  browsertechnologie als universele oplossing om de gebruikersinterfaces af te handelen.

\bigbreak{}

Er zijn veel opportuniteiten om kunstmatige intelligentie gebruiksvriendelijker te maken voor het bredere publiek en hierdoor de technologie op een grotere schaal te implementeren. Deze implementatie komt echter met een hogere kost voor de \textit{cloudproviders}, die gedragen zal worden door de eind\-ge\-brui\-ker~\autocite{Khan2024}. Het is niet onlogisch om hierbij de gelijkenis te trekken aan een vorige technologische revolutie. Namelijk die van het \textit{world wide web} waarbij uiteindelijk de eind\-ge\-brui\-ker het product werd en fungeert als het verdienmodel~\autocite{quoteresearch2017, OKO2019}.

\bigbreak{}

Om kunstmatige intelligentie beschikbaar te houden voor het grotere publiek en hierbij toch privacy en het milieu in acht te houden, moet er worden onderzocht hoe deze technologie kan worden ingezet op grote schaal.

\bigbreak{}

Niet enkel het trainen van kunstmatige intelligentie vergt veel rekenkracht, zo ook het operationeel houden hiervan~\autocite{Patel2023}. De kosten bij het implementeren van deze AI-modellen op grote schaal kunnen hierdoor sterk toenemen. Het opschalen van een nieuwe technologie gaat altijd gepaard met een financiële impact, maar binnen een markt waar er een tekort heerst voor computer componenten  kunnen deze kosten nog sterker oplopen. Een hoge prioriteit bij het implementeren van AI-modellen voor bedrijven bestaat uit het kostenefficiënt uitbouwen van capaciteiten waarbij de ecologische impact wordt geminimaliseerd.

\bigbreak{}

\section{Afbakenen van het onderwerp}

Onder andere \textit{Microsoft} maar ook veel andere bedrijven bieden hun AI-modellen als dienst aan op grote schaal. Omdat kunstmatige intelligentie nog in een vroeg stadium van ontwikkeling zit, is deze technologie nog niet volledig doorgestroomd naar kleinere bedrijven. Deze bedrijven zouden in de toekomst zelf deze technologie kunnen inzetten om bijvoorbeeld klan\-ten\-on\-der\-steu\-ning te voorzien. Op dit moment ontstaan nieuwe \textit{opensource} gemeenschappen rond kunstmatige intelligentie, zoals \href{https://huggingface.co/}{huggingface.co}. Het is niet onlogisch voor kleine softwarebedrijven om deze \textit{opensource} initiatieven verder uit te bouwen en hierdoor kunstmatige intelligentie vervolgens zelf in te zetten binnen reeds bestaande webtechnologieën zoals web\-app\-li\-ca\-ties en \textit{Progressive Web Apps}.

\bigbreak{}

Uiteraard moeten de initiële, maar ook de terugkerende kosten van het implementeren van kunstmatige intelligentie worden onderzocht, alvorens een bedrijf kan beginnen met een integratie. Hierbij kan men zich de vraag stellen hoe deze technologie op een zo kostenefficiënt mogelijke manier kan worden geïmplementeerd? Het antwoord hierop lijkt in de huidige markt bijna vanzelfsprekend. Er wordt namelijk snel gekozen voor \textit{Software as a Service} (\textit{Saas}), waarom zou dit voor het toepassen van kunstmatige intelligentie anders moeten?

\section{\IfLanguageName{dutch}{Probleemstelling}{Problem Statement}}%
\label{sec:probleemstelling}

% Uit je probleemstelling moet duidelijk zijn dat je onderzoek een meerwaarde heeft voor een concrete doelgroep. De doelgroep moet goed gedefinieerd en afgelijnd zijn. Doelgroepen als ``bedrijven,'' ``KMO's'', systeembeheerders, enz.~zijn nog te vaag. Als je een lijstje kan maken van de personen/organisaties die een meerwaarde zullen vinden in deze bachelorproef (dit is eigenlijk je steekproefkader), dan is dat een indicatie dat de doelgroep goed gedefinieerd is. Dit kan een enkel bedrijf zijn of zelfs één persoon (je co-promotor/opdrachtgever).

Met de globale introductie van kunstmatige intelligentie aan het bredere publiek werd al snel ondervonden dat het operationeel houden van deze AI-modellen op grote schaal gepaard gaat met enorme kosten~\autocite{Patel2023}.

\begin{displayquote}[\cite{Patel2023}]
    "Estimating ChatGPT costs is a tricky proposition due to several unknown variables. We built a cost model indicating that ChatGPT costs \$694.444 per day to operate in compute hardware costs."
\end{displayquote}

Het uitbesteden van rekenkracht is hierdoor relevanter dan ooit. Softwarebedrijven die kunstmatige intelligentie proberen te integreren in projecten beschikken over beperkte mogelijkheden om deze technologie te implementeren. De rekenkracht die vereist is om kunstmatige intelligentie op grote schaal te ondersteunen moet of worden uitbesteed, of lokaal worden opgebouwd. Een alternatieve oplossing hiervoor is echter deze berekeningen uit te voeren op apparaten van eindgebruikers. Dit is een van de potentiële toepassingen van \textit{WebGPU} die toelaat om rekenkundige taken lokaal uit te voeren in de browser op hardware van de eindgebruiker~\autocite{Wallez2023}.

\section{\IfLanguageName{dutch}{Onderzoeksvraag}{Research question}}%
\label{sec:onderzoeksvraag}

% Wees zo concreet mogelijk bij het formuleren van je onderzoeksvraag. Een onderzoeksvraag is trouwens iets waar nog niemand op dit moment een antwoord heeft (voor zover je kan nagaan). Het opzoeken van bestaande informatie (bv. ``welke tools bestaan er voor deze toepassing?'') is dus geen onderzoeksvraag. Je kan de onderzoeksvraag verder specifiëren in deelvragen. Bv.~als je onderzoek gaat over performantiemetingen, dan 

Is \textit{WebGPU} een geschikte technologie om de rekenkracht, die vereist is bij het operationeel inzetten van kunstmatige intelligentie, op grote schaal over te dragen aan de eindgebruiker? Staat deze overdracht ook een verbetering toe op vlak van gebruikservaring, privacy en prestaties? Worden processen zoals inferentie ondersteund, en kunnen geavanceerde AI-modellen hierdoor lokaal beschikbaar worden gesteld?

\section{\IfLanguageName{dutch}{Onderzoeksdoelstelling}{Research objective}}%
\label{sec:onderzoeksdoelstelling}

% Wat is het beoogde resultaat van je bachelorproef? Wat zijn de criteria voor succes? Beschrijf die zo concreet mogelijk. Gaat het bv.\ om een proof-of-concept, een prototype, een verslag met aanbevelingen, een vergelijkende studie, enz.

Binnen dit onderzoek worden de prestaties van \textit{WebGPU} vergeleken met voorgangers zoals \textit{WebGL}. Deze vergelijking wordt gedaan aan de hand van performantie testen voor zowel het trainen als het algemeen uitvoeren van AI-modellen. Ook wordt er onderzocht hoe het lokale installatieproces voor de eindgebruiker verschilt met technologieën zoals \textit{CUDA}. Uiteindelijk kan hierdoor een beter beeld geschetst worden hoe de gebruikerservaring al dan niet verbetert wanneer er gebruik wordt gemaakt van lokale kunstmatige intelligentie op browsers ondersteund door \textit{WebGPU}. Aan de hand van een prototype zal uiteindelijk worden aangewezen of het lokaal gebruik van kunstmatige intelligentie in de vorm van \textit{large language models} mogelijk is door middel van \textit{WebGPU}.

\bigbreak{}

De technologische capaciteiten van \textit{WebGPU} worden onderzocht evenals in hoeverre deze inzetbaar zijn voor het optimaliseren van kunstmatige intelligentie op het web. Hieruit volgt de conclusie dat \textit{WebGPU} al dan niet tot een revolutie zal leiden voor de lokale uitvoering van kunstmatige intelligentie binnen web\-app\-li\-ca\-ties. De \textit{Proof of Concept} laat toe dat de \textit{WebGPU} technologie makkelijk kan worden weergegeven en getest.

\bigbreak{}

Het succes van het onderzoek hangt sterk af van de huidige vooruitgang van deze nieuwe \textit{WebGPU} technologie. De resultaten van de testen en het prototype dat voor dit onderzoek werd uitgewerkt, kunnen, indien de technologie voldoende ver ontwikkeld is, worden gebruikt om toekomstig onderzoek te ondersteunen.

\section{\IfLanguageName{dutch}{Opzet van deze bachelorproef}{Structure of this bachelor thesis}}%
\label{sec:opzet-bachelorproef}

% Het is gebruikelijk aan het einde van de inleiding een overzicht te
% geven van de opbouw van de rest van de tekst. Deze sectie bevat al een aanzet
% die je kan aanvullen/aanpassen in functie van je eigen tekst.

Deze bachelorproef is als volgt opgebouwd:

\bigbreak{}

In hoofdstuk~\ref{ch:stand-van-zaken} wordt een overzicht gegeven van de stand van zaken binnen het onderzoeksdomein, op basis van een literatuurstudie.

\bigbreak{}

In hoofdstuk~\ref{ch:methodologie} wordt de methodologie toegelicht en worden de gebruikte onderzoekstechnieken besproken om een antwoord te kunnen formuleren op de onderzoeksvragen.

\bigbreak{}

In hoofdstuk~\ref{ch:technologylist} wordt een lijst vermeld en toegelicht van technologieën die reeds gebruik maken van \textit{WebGPU}.

\bigbreak{}

In hoofdstuk~\ref{ch:benchmarks} worden testresultaten geanalyseerd wat leidt tot een duidelijker beeld van de performantie van \textit{WebGPU}.

\bigbreak{}

In hoofdstuk~\ref{ch:poc} worden enkele technologieën besproken die werden geïmplementeerd als \textit{Proof of Concept}. Hierbij wordt er toegelicht welke resultaten verwacht kunnen worden van kunstmatige intelligentie uitgevoerd met \textit{WebGPU}.

\bigbreak{}

In hoofdstuk~\ref{ch:conclusie} wordt de conclusie gegeven en een antwoord geformuleerd op de onderzoeksvragen. Daarbij wordt ook een aanzet gegeven voor toekomstig onderzoek binnen dit domein.