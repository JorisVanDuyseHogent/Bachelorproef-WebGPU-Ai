%==============================================================================
% Sjabloon onderzoeksvoorstel bachproef
%==============================================================================
% Gebaseerd op document class `hogent-article'
% zie <https://github.com/HoGentTIN/latex-hogent-article>

% ss: voeg de documentclass-optie [english] toe.
% Let op: kan enkel na toestemming van de bachelorproefcoördinator!
\documentclass{hogent-article}
\usepackage{pgfgantt}
\usepackage{tikz}

% Invoegen bibliografiebestand
\addbibresource{voorstel.bib}

% Informatie over de opleiding, het vak en soort opdracht
\studyprogramme{Professionele bachelor toegepaste informatica}
\course{Bachelorproef}
\assignmenttype{Onderzoeksvoorstel}
% Voor een voorstel in het Engels, haal de volgende 3 regels uit commentaar
% \studyprogramme{Bachelor of applied information technology}
% \course{Bachelor thesis}
% \assignmenttype{Research proposal}

\academicyear{2023-2024}

\title{Kunstmatige intelligentie in web applicaties via WebGPU: onderzoek naar client-sided AI-modellen}

\author{Joris Van Duyse}
\email{joris.vanduyse@student.hogent.be}

% TODO: Geef de co-promotor op
% \supervisor[Co-promotor]{N/A (N/A, \href{mailto:email@example.com}{email@example.com})}

\projectrepo{https://github.com/JorisVanDuyseHogent/Bachelorproef-WebGPU-Ai}
\specialisation{Mobile \& Enterprise development}
\keywords{Web Development, WebGPU, Progressive Web Apps (PWA's), AI-modellen, Client-side, Webgebaseerde AI, Whisper-model, User Experience (UX), Privacy, Schaalbaarheid}

\begin{document}

\begin{abstract}
% Hier schrijf je de samenvatting van je voorstel, als een doorlopende tekst van één paragraaf. Let op: dit is geen inleiding, maar een samenvattende tekst van heel je voorstel met inleiding (voorstelling, kaderen thema), probleemstelling en centrale onderzoeksvraag, onderzoeksdoelstelling (wat zie je als het concrete resultaat van je bachelorproef?), voorgestelde methodologie, verwachte resultaten en meerwaarde van dit onderzoek (wat heeft de doelgroep aan het resultaat?).

In dit onderzoek wordt de integratie van kunstmatige intelligentie (AI) in \textit{Progressive Web Apps} (PWAs) door middel van WebGPU onderzocht. Deze integratie valt binnen het domein \textit{web development}, en heeft tot doel het lokaal uitvoeren van AI-modellen te verkennen door gebruik te maken van WebGPU.\@ Door AI-modellen rechtstreeks in de browser te laten draaien wordt het installatieproces vereenvoudigd. Hierbij ligt de focus op de implementatie van AI-modellen, zoals onder andere het Whisper-model. Er wordt ook specifiek aandacht besteed aan potentiële voordelen op het gebied van prestaties en gebruikerservaring. Verwachte resultaten omvatten een vergelijkende analyse van de prestaties van WebGPU met bestaande renderer technologieën, zoals WebGL en CUDA. Dit onderzoek draagt bij aan de wetenschappelijke kennis over de synergie tussen AI, webtechnologieën en grafische rendering, met implicaties voor ontwikkelaars en professionals die betrokken zijn bij de integratie van AI in webomgevingen.
\end{abstract}

\tableofcontents

% De hoofdtekst van het voorstel zit in een apart bestand, zodat het makkelijk
% kan opgenomen worden in de bijlagen van de bachelorproef zelf.
%---------- Inleiding ---------------------------------------------------------

\section{Introductie}%
\label{sec:introductie}

% Waarover zal je bachelorproef gaan? Introduceer het thema en zorg dat volgende zaken zeker duidelijk aanwezig zijn:

\begin{itemize}
  \item kaderen thema
  \item de doelgroep
  \item de probleemstelling en (centrale) onderzoeksvraag
  \item de onderzoeksdoelstelling
\end{itemize}

% Denk er aan: een typische bachelorproef is \textit{toegepast onderzoek}, wat betekent dat je start vanuit een concrete probleemsituatie in bedrijfscontext, een \textbf{casus}. Het is belangrijk om je onderwerp goed af te bakenen: je gaat voor die \textit{ene specifieke probleemsituatie} op zoek naar een goede oplossing, op basis van de huidige kennis in het vakgebied.

% De doelgroep moet ook concreet en duidelijk zijn, dus geen algemene of vaag gedefinieerde groepen zoals \emph{bedrijven}, \emph{developers}, \emph{Vlamingen}, enz. Je richt je in elk geval op it-professionals, een bachelorproef is geen populariserende tekst. Eén specifiek bedrijf (die te maken hebben met een concrete probleemsituatie) is dus beter dan \emph{bedrijven} in het algemeen.

% Formuleer duidelijk de onderzoeksvraag! De begeleiders lezen nog steeds te veel voorstellen waarin we geen onderzoeksvraag terugvinden.

% Schrijf ook iets over de doelstelling. Wat zie je als het concrete eindresultaat van je onderzoek, naast de uitgeschreven scriptie? Is het een proof-of-concept, een rapport met aanbevelingen, \ldots Met welk eindresultaat kan je je bachelorproef als een succes beschouwen?

Dit onderzoek verkent de grenzen van webtechnologieën door zich te richten op de integratie van opensource kunstmatige intelligentie (AI) modellen 
met WebGPU in Progressive Web Apps (PWAs). 
De snelle evolutie van webtechnologieën, zoals beschreven door \textcite{Shumylo2023}, 
biedt nieuwe mogelijkheden voor het uitvoeren van complexe taken rechtstreeks in de browser.
In het specifieke domein van AI en opensource-modellen presenteert dit onderzoek een innovatieve benadering 
waarbij WebGPU wordt ingezet voor de lokale uitvoering van AI-modellen in de browser.

\bigbreak

Deze benadering gaat verder dan de traditionele verschuiving van rekenkracht van servers naar de client-side. 
Het legt de nadruk op de adaptatie aan de hedendaagse trend van Progressive Web Apps (PWA), 
waarmee de gebruikerservaring aanzienlijk kan worden verbeterd door de installatie van AI-modellen te vereenvoudigen. 
Na het opzetten van de proof-of-concept zal een vergelijking plaatsvinden op het gebied van performantie tussen WebGPU en een uitvoering van een model met CUDA, 
waarmee inzichten worden verkregen in de efficiëntie van deze benaderingen.

\bigbreak

Tevens richt dit onderzoek zich op de gebruiksvriendelijkheid van opensource AI-modellen binnen PWAs, 
waarbij de combinatie van PWA en WebGPU wordt onderzocht om het installatieproces van deze modellen te stroomlijnen en toegankelijk te maken voor een breder publiek. 
De studie omvat ook een grondige analyse van veiligheidsaspecten, waarbij de verschillen tussen WebGL en WebGPU met betrekking tot de toegang tot hardware grafische kaarten worden onderzocht. 
Deze inzichten dragen bij aan het begrip van zowel de mogelijkheden als de beperkingen van geavanceerde webtechnologieën in het domein van AI-integratie.


%---------- Stand van zaken ---------------------------------------------------

\section{State-of-the-art}%
\label{sec:state-of-the-art}

% Hier beschrijf je de \emph{state-of-the-art} rondom je gekozen onderzoeksdomein, d.w.z.\ een inleidende, doorlopende tekst over het onderzoeksdomein van je bachelorproef. Je steunt daarbij heel sterk op de professionele \emph{vakliteratuur}, en niet zozeer op populariserende teksten voor een breed publiek. Wat is de huidige stand van zaken in dit domein, en wat zijn nog eventuele open vragen (die misschien de aanleiding waren tot je onderzoeksvraag!)?

% Je mag de titel van deze sectie ook aanpassen (literatuurstudie, stand van zaken, enz.). Zijn er al gelijkaardige onderzoeken gevoerd? Wat concluderen ze? Wat is het verschil met jouw onderzoek?

% Verwijs bij elke introductie van een term of bewering over het domein naar de vakliteratuur, bijvoorbeeld~\autocite{Hykes2013}! Denk zeker goed na welke werken je refereert en waarom.

% Draag zorg voor correcte literatuurverwijzingen! Een bronvermelding hoort thuis \emph{binnen} de zin waar je je op die bron baseert, dus niet er buiten! Maak meteen een verwijzing als je gebruik maakt van een bron. Doe dit dus \emph{niet} aan het einde van een lange paragraaf. Baseer nooit teveel aansluitende tekst op eenzelfde bron.

% Als je informatie over bronnen verzamelt in JabRef, zorg er dan voor dat alle nodige info aanwezig is om de bron terug te vinden (zoals uitvoerig besproken in de lessen Research Methods).

% Voor literatuurverwijzingen zijn er twee belangrijke commando's:
% \autocite{KEY} => (Auteur, jaartal) Gebruik dit als de naam van de auteur
%   geen onderdeel is van de zin.
% \textcite{KEY} => Auteur (jaartal)  Gebruik dit als de auteursnaam wel een
%   functie heeft in de zin (bv. ``Uit onderzoek door Doll & Hill (1954) bleek
%   ...'')

% Je mag deze sectie nog verder onderverdelen in subsecties als dit de structuur van de tekst kan verduidelijken.

Als onderdeel van mijn onderzoek omtrent de integratie van opensource kunstmatige intelligentie (AI) 
modellen met WebGPU binnen Progressive Web Apps (PWAs), 
wordt de aanzienlijke waarde van PWAs als een moderne paradigmaverschuiving binnen webapplicaties benadrukt. 
De PWAs, zoals gepresenteerd door \textcite{Shumylo2023}, 
zijn zorgvuldig vormgegeven om een ervaring te bieden die vergelijkbaar is met native apps, 
gebruikmakend van geavanceerde webtechnologieën zoals Service Workers, Web App Manifest, en Push Notifications.

Deze technologische pijlers empoweren PWAs om functionaliteiten zoals offline-ondersteuning, 
snelle laadtijden, en pushmeldingen te integreren, 
waarmee zij een aantrekkelijke keuze worden voor gebruikers die streven naar een naadloze en betrokken mobiele interactie.

In tegenstelling tot traditionele webapplicaties vertoont het PWA-concept een reeks significante voordelen, 
waaronder versnelde laadtijden, 
robuuste offline-ondersteuning, een ervaring die parallel loopt aan native apps, 
grensoverschrijdende compatibiliteit, en moeiteloze ontdekbaarheid. 
Deze eigenschappen maken PWAs bijzonder geschikt als omgeving voor de uitvoering van complexe AI-taken aan de client-zijde, 
met de ondersteuning van WebGPU.

Dit onderzoek richt zich op een grondige verkenning van de mogelijkheden om opensource AI-modellen te integreren in deze vooruitstrevende webtoepassingen, 
met als doel de algehele gebruikerservaring te versterken en innovatieve inzichten te verschaffen binnen het domein van Mobile en Enterprise Development.


%---------- Methodologie ------------------------------------------------------
\section{Methodologie}%
\label{sec:methodologie}

% Hier beschrijf je hoe je van plan bent het onderzoek te voeren. Welke onderzoekstechniek ga je toepassen om elk van je onderzoeksvragen te beantwoorden? Gebruik je hiervoor literatuurstudie, interviews met belanghebbenden (bv.~voor requirements-analyse), experimenten, simulaties, vergelijkende studie, risico-analyse, PoC, \ldots?

% Valt je onderwerp onder één van de typische soorten bachelorproeven die besproken zijn in de lessen Research Methods (bv.\ vergelijkende studie of risico-analyse)? Zorg er dan ook voor dat we duidelijk de verschillende stappen terug vinden die we verwachten in dit soort onderzoek!

% Vermijd onderzoekstechnieken die geen objectieve, meetbare resultaten kunnen opleveren. Enquêtes, bijvoorbeeld, zijn voor een bachelorproef informatica meestal \textbf{niet geschikt}. De antwoorden zijn eerder meningen dan feiten en in de praktijk blijkt het ook bijzonder moeilijk om voldoende respondenten te vinden. Studenten die een enquête willen voeren, hebben meestal ook geen goede definitie van de populatie, waardoor ook niet kan aangetoond worden dat eventuele resultaten representatief zijn.

% Uit dit onderdeel moet duidelijk naar voor komen dat je bachelorproef ook technisch voldoen\-de diepgang zal bevatten. Het zou niet kloppen als een bachelorproef informatica ook door bv.\ een student marketing zou kunnen uitgevoerd worden.

% Je beschrijft ook al welke tools (hardware, software, diensten, \ldots) je denkt hiervoor te gebruiken of te ontwikkelen.

% Probeer ook een tijdschatting te maken. Hoe lang zal je met elke fase van je onderzoek bezig zijn en wat zijn de concrete \emph{deliverables} in elke fase?

\subsection*{Fase 1: WebGL en WebGPU}

\begin{itemize}
\item \textbf{Doelstelling}: Onderzoek naar de vershillen tussen WebGL en WebGPU
\item \textbf{Aanpak}:
\begin{itemize}
\item Wat bracht WebGL en wat ontbreekt het.
\item Wat is WebGPU en wat zijn de voordelen.
\item Hoe kan WebGPU worden gebruikt in een PWA.
\end{itemize}
\item \textbf{Tijdskader}: 2 weken
\item \textbf{Deliverable}: Een overzicht van de verschillen tussen WebGL en WebGPU.
\end{itemize}

Er wordt onderzocht in hoever WebGPU een verbetering is ten opzichte van WebGL. De performantie van beide technologieën wordt vergeleken. 
Er wordt ook onderzocht welke technologieën er nodig zijn om WebGPU te gebruiken en hoe deze verschillen met WebGL.

2 Weken

\subsection*{Fase 2: WebGPU, privacy en veiligheid}

\begin{itemize}
\item \textbf{Doelstelling}: Veiligheidsrisico analyse van WebGPU
\item \textbf{Aanpak}:
\begin{itemize}
\item 
\item 
\item 
\end{itemize}
\item \textbf{Tijdskader}: 2 weken
\item \textbf{Deliverable}: Een analyse van de veiligheidsrisico's van WebGPU.
\end{itemize}

\subsection*{Fase 3: Proof of concept, Ai en WebGPU}

\begin{itemize}
\item \textbf{Doelstelling}: Ai modellen integreren in een PWA met behulp van WebGPU
\item \textbf{Aanpak}:
\begin{itemize}
\item Een geschikt Ai model zoeken.
\item Een simpele PWA opzetten.
\item WebGPU integreren in de PWA.
\end{itemize}
\item \textbf{Tijdskader}: 6 weken
\item \textbf{Deliverable}: Een proof of concept van een Ai model in een PWA met behulp van WebGPU.
\end{itemize}
Er wordt onderzocht in hoeverre het Whisper AI model kan geïmplementeerd worden in een PWA met behulp van WebGPU.

\subsection*{Fase 4: Performantie WebGPU}

\begin{itemize}
\item \textbf{Doelstelling}: Performantie van WebGPU vergelijken met CUDA
\item \textbf{Aanpak}:
\begin{itemize}
\item Proof of concept testen aan de hand van varschillende benchmark.
\item Een CUDA implementatie van het Whisper AI model opzetten.
\item Vergelijken van de performantie van beide implementaties.
\end{itemize}
\item \textbf{Tijdskader}: 2 weken
\item \textbf{Deliverable}: Een vergelijking van de performantie van WebGPU met CUDA.
\end{itemize}

Na het uitwerken van een proof of concept kan de performantie van WebGPU vergeleken worden met alternatieve technologieën zoals CUDA.
Er zijn namelijk meerdere AI-modellen die gebruik maken van CUDA, zoals Whisper AI van OpenAI.

%---------- Verwachte resultaten ----------------------------------------------
\section{Verwacht resultaat, conclusie}%
\label{sec:verwachte_resultaten}

% Hier beschrijf je welke resultaten je verwacht. Als je metingen en simulaties uitvoert, kan je hier al mock-ups maken van de grafieken samen met de verwachte conclusies. Benoem zeker al je assen en de onderdelen van de grafiek die je gaat gebruiken. Dit zorgt ervoor dat je concreet weet welk soort data je moet verzamelen en hoe je die moet meten.

% Wat heeft de doelgroep van je onderzoek aan het resultaat? Op welke manier zorgt jouw bachelorproef voor een meerwaarde?

% Hier beschrijf je wat je verwacht uit je onderzoek, met de motivatie waarom. Het is \textbf{niet} erg indien uit je onderzoek andere resultaten en conclusies vloeien dan dat je hier beschrijft: het is dan juist interessant om te onderzoeken waarom jouw hypothesen niet overeenkomen met de resultaten.

De implementatie van complexe AI-modellen, zoals Whisper AI en Midjourney, 
in een webomgeving met behulp van WebGPU zal ongetwijfeld een uitdagend proces zijn. 
Het streven naar succes in dit onderzoek manifesteert zich niet alleen in de succesvolle integratie van deze modellen in Progressive Web Apps (PWA's), 
maar ook in het bereiken van prestaties op het niveau van gevestigde systemen, zoals Whisper AI met CUDA. 
De complexiteit van deze taak wordt benadrukt door de noodzaak om de rekenkracht van WebGPU te optimaliseren, 
waarbij het doel is om vergelijkbare prestaties te behalen als die welke worden geboden door meer traditionele uitvoeringsomgevingen. 
Het succes van dit onderzoek zal niet alleen worden afgemeten aan de volledige en stabiele werking van de geïmplementeerde AI-modellen op de client-side, 
maar ook aan de mate waarin WebGPU een gelijkwaardige of zelfs verbeterde prestatie kan leveren in vergelijking met CUDA. 
Het streven naar deze prestatie-equivalentie met gevestigde technologieën markeert een significante mijlpaal 
en draagt bij aan het begrip van de mogelijkheden van WebGPU voor het uitvoeren van veeleisende AI-taken binnen webomgevingen.


\printbibliography[heading=bibintoc]

\clearpage


\section{Gantt chart}%
\begin{ganttchart}[
  x unit=0.6cm,
  y unit title=0.6cm,
  y unit chart=0.8cm,
  vgrid,hgrid,
  title label anchor/.style={below=-1.6ex},
  title left shift=.05,
  title right shift=-.05,
  title height=1,
  progress label text={},
  bar height=0.5,
  group right shift=0,
  group top shift=.6,
  group height=.3,
  group peaks tip position=0
  ]{1}{16}
%labels
\gantttitle{2024}{16} \\
\gantttitle{Feb}{4}
\gantttitle{Mar}{4}
\gantttitle{Apr}{4}
\gantttitle{May}{4}\\

%tasks
\ganttgroup{Fase 1: Literatuurstudie}{1}{2} \\ %elem0
\ganttbar{Beoordeling van literatuur}{1}{1} \\ %elem1
\ganttbar{Analyse praktijkvoorbeelden}{1}{1} \\ %elem2
\ganttbar{Identificatie benodigde technologieën}{2}{2} \\ %elem3
\ganttmilestone{Fase 1 Afgerond}{2} \\ %elem4

\ganttgroup{Fase 2: Requirement analyse}{3}{3} \\ %elem5
\ganttbar{FR en NFR bepalen}{3}{3} \\ %elem6
\ganttmilestone{Fase 2 Afgerond}{3} \\ %elem7

\ganttgroup{Fase 3: Long list}{4}{5} \\ %elem8
\ganttbar{Relevante framworks}{4}{4} \\ %elem9
\ganttbar{Geschikte AI-modellen}{5}{5} \\ %elem10
\ganttmilestone{Fase 3 Afgerond}{5} \\ %elem11

\ganttgroup{Fase 4: Short list}{6}{6} \\ %elem12
\ganttbar{Analyse long list}{6}{6} \\ %elem13
\ganttbar{Selectie AI-modellen en framworks}{6}{6} \\ %elem14
\ganttmilestone{Fase 4 Afgerond}{6} \\ %elem15

\ganttgroup{Fase 5: Proof of Concept}{7}{12} \\ %elem16
\ganttbar{Basisstructuur webapplicatie}{7}{8} \\ %elem17
\ganttbar{Implementatie WebGPU}{8}{10} \\ %elem18
\ganttbar{Implementatie AI-modellen}{9}{12} \\ %elem19
\ganttmilestone{Fase 5 Afgerond}{12} \\ %elem20

\ganttgroup{Fase 6: Conclusie}{13}{15} \\ %elem21
\ganttbar{Perstaties testen}{13}{14} \\ %elem22
\ganttbar{Vergelijking van de resultaten}{15}{15} \\ %elem23
\ganttmilestone{Fase 5 Afgerond}{15} \\ %elem24

\ganttbar{Finalisatie}{16}{16} \\ %elem25

\ganttmilestone{Eindoplevering}{16} \\ %elem26


\ganttlink{elem4}{elem5}
\ganttlink{elem7}{elem8}
\ganttlink{elem11}{elem12}
\ganttlink{elem15}{elem16}
\ganttlink{elem20}{elem21}
\ganttlink{elem22}{elem23}
\ganttlink{elem24}{elem25}

\end{ganttchart}
\end{document}